\documentclass[a4paper]{exam}

\usepackage{amsmath,amssymb, amsthm}
\usepackage{geometry}
\usepackage{graphicx}
\usepackage{hyperref}
\usepackage{xcolor}
\usepackage{wasysym}

\usepackage{tikz}
\usepackage{tikz-qtree}

\usepackage{scalerel}

\newcommand{\X}{\scalerel*{\includegraphics{Annihilus.png}}{\mathbf{0}}}



\newtheorem{definition}{Definition}
\newtheorem{theorem}{Theorem}

\title{Weekly Challenge 02: Deterministic Finite Automata}
\author{CS 212 Nature of Computation\\Habib University}
\date{Fall 2025}

\boxedpoints


% \printanswers % Uncomment this line

\begin{document}
\maketitle


\begin{questions}

  
\question
For a language $L \subseteq \Sigma^*$, the $\X$ operator is defined as follows: $\X(L) = \{w \in \Sigma^*\mid w \not\in L\}$. 

Consider the language $L$ of strings of form $0^n1^m$ for $n,m \in \mathbb{Z}^+$. 
\begin{parts}
    \part Define $\X(L)$, and construct a FSM (DFA) for $\X(L)$.
    \begin{solution}
        % Enter your solution here 
    \end{solution}

    \part Prove or disprove the following claim: If a language $L \subseteq \Sigma^*$ is regular then $\X(L)$ is also regular. 
    \begin{solution}
        % Enter your solution here 
    \end{solution}
\end{parts}


\end{questions}
\end{document}

%%% Local Variables:
%%% mode: latex
%%% TeX-master: t
%%% End:



